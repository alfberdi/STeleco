\subsection{Ejercicios}
\label{sub:ejercicios4}
\begin{exercise}[1]
	Un sistema celular formado por clusters de 4 celdas hexagonales de 1.38 km de radio, dispone de 60 canales. El tráfico ofrecido por cada usuario es de 0.029 E, que corresponden a una llamada en media por HC. Se trata de un sistema dotado de colas de espera, con objetivo de que la probabilidad de que un usuario tenga que esperar sea menor o igual que el 5\%. Se pregunta: \\
	\textbf{1.} ¿Cuántos usuarios por km2 puede soportar el sistema? \\
	Suponiendo celdas unisectoriales, podemos ver que cada celda tendrá asignados $\frac{c}{J}=\frac{60}{4}=15\sfrac{canales}{celda}$ con este dato y la probabilidad de espera, $P_B=0,05$, se puede obtener el tráfico ofrecido a la celda gracias a la función Erlang C. 
	\[C(c,A_o)=0,05\to A_o=8E\]
	Con estos datos y sabiendo que el tráfico por usuario es $a_o=0.029E$ obtenemos:
	\[N_{celda}=\frac{A_o}{a_o}=\frac{8}{0,029}=275\sfrac{usuarios}{celda}\]
	Sabiendo los usuarios por celda y el tamaño de la celda, la densidad de usuarios tendrá que ser menor que D para permanecer dentro de las especificaciones del sistema:
	\[D=\frac{N_{celda}}{S_{celda}}=\frac{275}{4.95km^2}=55,58\sfrac{usuarios}{km^2}\]
	\textbf{2.} ¿Cuál es la probabilidad de que una llamada espere más de 10 segundos?
	\[a_o=\frac{\lambda}{\mu}\to \mu=\frac{\lambda}{a_o}=\frac{1}{0,029}=34,48\sfrac{llamadas}{HC}=0.01\sfrac{llamadas}{s}\]
	\[P(t\geq 10s)=1-e^{-\mu 10}=0,09\]
\end{exercise}
\begin{exercise}[2]
	Un sistema PMR de asignación troncal dispone de las siguientes bandas de frecuencia, dentro de la banda asignada: 380-382/390-392MHz, con canalización de 25kHz. Sobre cada portadora se establece una trama con cuatro intervalos de tiempo, sobre la que se realizará un acceso múltiple TDMA/FDMA. La velocidad binaria en la interfaz radio es de 30 kbps. Con este sistema se pretende prestar un servicio de gestión de flotas a una compañía que opera en el centro de Madrid, constituida por un total de 100 vehículos. Se estima que la gestión de la flota requiere cursar un tráfico de 0.7E en sentido despacho$\to$flota y 0.3E en sentido flota$\to$despacho, ambos en la hora cargada (utilizando canales de 7.5kbps en cada sentido). Se requiere operación full-duplex. Se pregunta:\\
	\textbf{1.} ¿Cuántas portadoras es necesario habilitar para obtener un grado de servicio del 0.1 \%? Elija los valores de frecuencia\\
	%Como la operación ha de ser full-duplex, y hay dos bandas bien diferenciadas, usaremos FDD para la duplexación.
	%\[c=\frac{BW}{BW_{porta}}=\frac{2MHz}{25\sfrac{kHz}{portadora}}=80\text{ portadoras full duplex}\]
	En este problema hay dos soluciones bien diferentes basadas en dos suposiciones: 
	\begin{enumerate}
		\item Si suponemos Que los tráficos del enunciado son por vehículo 
		\item Si en cambio los suponemos tráficos totales
	\end{enumerate}
	Si suponemos lo primero:
	\begin{gather*}
		a_o=0,7+0,3=1E\\
		A_o=N*a_o=100E\\
		GoS=0,001=C(c,A_o)\to c\approx 200canales\\
		\Delta f=c\frac{1}{4\sfrac{canales}{portadora}}25\sfrac{kHz}{portadora}=1,25MHz
	\end{gather*}
	En este caso las bandas utilizadas serían:380-381,25/390-391,25MHz.\\
	Si en cambio suponemos tráficos totales:
	\begin{gather*}
		A_o=0,7+0,3=1E\\
		GoS=0,001=C(c,A_o)\to c\approx 6canales\\
		\Delta f=c\frac{1}{4\sfrac{canales}{portadora}}25\sfrac{kHz}{portadora}=50kHz
	\end{gather*}
	En este caso las bandas utilizadas serían:380-380,05/390-390,05MHz.\\
	\textbf{2.} Obtenga el valor de la eficiencia de trunking (número de usuarios por canal)\\
	Estamos en el mismo problema que antes, hay dos casos.\\
	En el caso de ser tráfico por usuario:
	\[\eta=\frac{N}{c}=\frac{100}{200}=0,5\sfrac{usuarios}{canal}\]
	En el caso de ser tráficos totales:
	\[\eta=\frac{N}{c}=\frac{100}{6}=16,67\sfrac{usuarios}{canal}\]
	\textbf{3.} ¿Qué grado de servicio se obtendría si se utilizara asignación fija con el mismo número de usuarios por canal?
	Estamos en el mismo problema que antes, hay dos casos.\\
	En el caso de ser tráfico por usuario:
	\begin{gather*}
		A_o=a_o \eta=0.5E\\
		GoS=0,5
	\end{gather*}
	Como se puede ver al pasar de trunking a asignación fija el grado de servicio baja al 50\%.\\
	En el caso de ser tráficos totales:
	\begin{gather*}
		a_o=\frac{A_o}{N}=0.01E
		A_o=a_o \eta=0.1667E\\
		GoS=0,1667
	\end{gather*}
	Como se puede ver al pasar de trunking a asignación fija el grado de servicio baja al 16,67\%, no se ve una mejora tan inmensa como en el otro caso, pero sigue siendo considerable.\\
\end{exercise}
\begin{exercise}[3]
	Se desea competir por una licencia de telefonía celular en una ciudad de 4 millones de habitantes. El requisito principal es ofrecer servicio a un 20\% de la población con una probabilidad de bloqueo del 2\%. El ancho de banda asignado al servicio es de 4 MHz (a compartir en FDD) y se va a utilizar una modulación que requiere una relación de protección (S/I) de al menos 13 dB para funcionar correctamente en canales de 20 kHz. Se ha realizado un estudio de mercado y se prevé un perfil de usuarios que realizan una llamada de 2 minutos de duración cada hora. Estudiar cuál es el diseño más adecuado y el coste de la inversión del primer año, teniendo en cuanta que se estima un coste de los equipos de 48000€ por estación base y un precio medio del alquiler de los emplazamientos de 12000€ por estación base al año.\\
	\begin{gather*}
		P_B=2\%\\
		c=\frac{BW}{BW_{canal}}=\frac{2MHz}{20\sfrac{kHz}{canal}}=100canales\\
		a_o=\frac{\lambda}{\mu}=1\sfrac{llamada}{HC}\frac{1HC}{60min}2\sfrac{min}{llamada}=\frac{1}{30}E\\
		\frac{S}{I}=10log(\frac{(\sqrt{3J})^n}{i_0})=13dB\to J=(10^{\frac{S}{I}\frac{1}{10}}i_0)^{\frac{2}{n}}\frac{1}{3}\\
		c_{sector}\leq\frac{c}{J*J_{\sfrac{sector}{celda}}}\\
		A_{osector}=B^{-1}(c,P_B)\\
		N_{sector}\leq\frac{A_{osector}}{a_o}\\
		N_{celda}=N_{sector}J_{\sfrac{sector}{celda}}\\
		N_{cluster}=N_{celda}*J\\
		M_{cluster}\geq\frac{N}{N_{cluster}}\\
		M_{BTS}=M_{cluster}J\\
		Precio=M_{BTS}*(48000+12000)
	\end{gather*}
	\begin{center}
	\begin{tabular}{l|c|c|c|c|c|c|c|c}
		Sectorización  	& $i_0$	& J & $c_{sector}$  & $A_{osector}$ & $N_{cluster}$ & $M_{cluster}$ 	& $M_{BTS}$ & Precio(M€) 	\\\hline
		No sectorizada 	& 6	& 4 & 25	& 16	& 1920	& 417 & 1668 & 100,08	\\\hline
		2 sectores		& 3	& 3 & 16	& 10	& 1800	& 445 & 1335 & 80,1		\\\hline
		3 sectores		& 2	& 3 & 11	& 7.5	& 2025	& 396 & 1188 & 71,28	\\\hline
		6 sectores		& 1	& 3 & 5		& 1.8	& 972	& 823 & 2469 & 148,14	\\\hline
	\end{tabular}
	\end{center}
	De la tabla anterior podemos ver que la mejor opción es la instalación de antenas de tres sectores.
\end{exercise}
\begin{exercise}[4]
	Un sistema celular digital de telefonía móvil pública (PMT) que opera en una frecuencia de 900 MHz desea establecer una estación base en el interior de un centro comercial (entorno indoors LOS). La PIRE de dicha estación base es de 100W. Se desea estimar el tamaño de la zona de cobertura de esta célula. Para ello se ha especificado que un móvil situado en el borde de esta zona con una antena de ganancia 0 dB debe recibir una potencia superior a 10 dBm durante el 99\% del tiempo.\\
	\textbf{NOTA:} Se ha medido las pérdidas de propagación a 1 m de la antena de la estación base y se ha obtenido un valor medio de 1 dB. El modelo log-normal tiene una expresión para las pérdidas de la siguiente forma:
	\[L(d)=L(d_0)+10nlog(\frac{d}{d_0})\]
	Si el valor de las pérdidas sufre variaciones entorno a la media de tipo log-normal, con desviación estándar $\sigma$, la probabilidad de que la potencia recibida sea superior a un valor $\gamma$ dado es: 
	\[P(\overline{P_{rx}}\geq \gamma)=Q(\frac{\gamma-\overline{P_{rx}}}{\sigma})\]
	siendo $\overline{P_{rx}}$ el valor medio de la potencia recibida\\
	\textbf{1.} Calcule el radio de la zona de cobertura. Utilice el modelo log-normal con n=2,2 y considere que la desviación estándar de las pérdidas de propagación en este entorno es de $\sigma$=10dB.\\
	\begin{gather*}
		P_{rx}=P_{tx}+G_{tx}+G_{rx}-L\\
		L(d)=L(d_0)+10nlog(\frac{d}{d_0})\\
		P(\overline{P_{rx}}\geq 10dBm)=Q(\frac{\gamma-\overline{P_{rx}}}{\sigma})=0.99\\
		Q(-x)=1-Q(x)\to x=-2.33\\
		\frac{\gamma-\overline{P_{rx}}}{\sigma}=2.33\to \overline{P_{rx}}\geq 33.3dBm\\
		\overline{P_{rx}}\geq 33.3dBm=PIRE-L\to L\leq 13.3dBm\\
		R_{max}=5.17
	\end{gather*}
	\textbf{2.} Se desea transmitir datos 1 Mbps utilizando una modulación GMSK-0.25. Considerando que la potencia recibida por el móvil en el borde de la zona de cobertura es la que se especificó para el 99\% del tiempo (10 dBm) y que la conversión de RF a banda base en el receptor del móvil no introduce pérdidas, calcule la relación $\sfrac{E_b}{N_0}$, sabiendo que la potencia de ruido medida en la banda base es de $5*10^{-4}$W\\
	\begin{gather*}
		N_0=5*10^{-4}W\\
		\frac{E_b}{N_0}=10log(\frac{S}{N}BT)=7dB
	\end{gather*}
	\textbf{3.} ¿Qué probabilidad de error (BER) se obtiene en estas condiciones? \\
	\[BER=Q(\sqrt{2\gamma\frac{E_b}{N_0}})=Q(2.61)=4.5*10^{-3}\]
	\textbf{4.} El sistema utiliza un esquema de multiacceso FDMA/FDD en la banda de frecuencias de 890-915MHz MS$\to$BS, y 935-960MHz BS$\to$MS. si no se dejan frecuencias de guarda en los extremos de la banda, ¿de cuántos canales duplex se dispone?\\
	Si las dos bandas no fuesen iguales sería más beneficioso usar TDD que FDD ya que se podría utilizar todas las frecuencias. En este caso al ser las dos bandas iguales da igual que método de duplexación apliquemos.
	\begin{gather*}
		B=R_bBT=250\sfrac{kHz}{canal}\\
		c=\frac{25MHz}{250\sfrac{kHz}{canal}}=100\text{ canales duplex}
	\end{gather*}
	\textbf{5.} Esta estación base tiene adjudicados el 10\% de los canales disponibles. Se estima que los móviles van a realizar una llamada por hora cargada, con una duración media de 1 minuto y se desea obtener un grado de servicio GoS del 2\%. ¿A cuántos móviles puede atender la estación base?\\
	\begin{gather*}
	\lambda=1\sfrac{llamada}{HC}\\
	E(s)=1\sfrac{minuto}{llamada}\\
	a_o=E(s)\lambda=0.0167E\\
	GoS=P_B=2\%=B(c,A_o)\\
	A_o=5E\\
	M_{BS}=\frac{A_o}{a_o}=\frac{5}{0.0167}=300\sfrac{usuarios}{BS}
\end{gather*}
\end{exercise}
\begin{exercise}[5]
	Trabaja en una consultora a la que le han pedido que calcule el grado de servicio en una zona urbana (índice de pérdidas igual a 4) de un operador de telefonía GSM que dispone de 42 pares de portadoras para ofrecer sus servicios. Para que el sistema funcione correctamente es necesario una relación de protección de 15 dB. Las celdas son hexagonales y están sectorizadas en 3 sectores. El radio de la celda es de 0.5 km. Para calcular el número de usuarios por celda no se va a tener en cuenta la densidad de la población, sino algo más realista como es la capacidad máxima de señalización de las estaciones base. Como sabe, el proceso de petición de llamadas se puede considerar que está basado en Aloha ranurado, en el que la capacidad es de 1.28 kbps y en el que cada intento produce un mensaje de 160 bits. Asuma que la eficiencia es la máxima posible, aunque luego solo tenga en cuenta el 10\% de los usuarios obtenidos dado que existen limitaciones en los canales de tráfico de las estaciones base. Se sabe que los usuarios realizan 2 llamadas en la hora cargada y además reciben 1 llamada en la hora cargada. También, que la probabilidad de que una llamada supere los 3 minutos es de 0,04978.\\ 
	\textbf{CONSEJO:} Calcule el número de usuarios máximo por estación base. \\
	\textbf{NOTA:} Asuma un canal de control por sector.\\
	Para el cálculo del número de usuarios analizaremos el sistema ALOHA ranurado que se usa en el canal de control:
	\begin{gather*}
		S=I=0.36\\
		a_0=\frac{160bits*3\sfrac{llamadas}{HC}}{3600\sfrac{s}{HC}}=\frac{2}{15}E\\
		C=1.28kbps\\
		N'_{sector}=\frac{I*C}{a_0}=3456usuarios
	\end{gather*}
	Como Se pide que solo se tenga el 10\% de los usarios posibles $N_{sector}=345$. 
	\begin{gather*}
		\frac{S}{I}=10log(\frac{(\sqrt{3J})^n}{i_0})\geq 15dB\\
		J=7\sfrac{celdas}{cluster}\\
		\frac{42portadoras}{J}\frac{1celda}{3sectores}8\sfrac{canales}{portadora}-1=15\sfrac{canales}{sector}\\
		P(t\geq 3min)=1-e^{-\mu 3min}=0.04978 \to \mu=1.021\sfrac{llamadas}{min}\\
		a_o=\frac{\lambda}{\mu}=\frac{3\sfrac{llamada}{HC}}{1.021\sfrac{llamadas}{min}}\frac{1}{60}=0.05E\\
		A_o=a_o N_{sector}=17.25E\\
		GoS=B(c,A_o)=0,25
	\end{gather*}
	hola
\end{exercise}
\begin{exercise}[7]
	La estación base de una celda de 50 km de radio situada en un entorno suburbano de un sistema celular analógico que funciona en una frecuencia de 900 MHz transmite una PIRE de 1 kW. La antena de dicha estación está situada a una altura de 100 m. Se desea estudiar la calidad del enlace para un móvil situado en en borde de la zona de cobertura de esta estación. Su antena está situada a una altura de 10 m y la ganancia del terminal es de 0 dB.\\ 
	NOTA: Fórmulas de Okumura-Hata. Expresión del valor mediano de las pérdidas según Okumura-Hata (150 MHz - 1500 MHz):
	\begin{gather*}
		L_{50}(dB)=L_{50urban}(dB)-2(log(\frac{f_c}{28}))^2-5.4\\
		L_{50urban}(dB)=69.55+26.16log(f_c)-13.82log(h_{tx})-a(h_{rx})+(44.9-6.55log(h_{tx}))log(d)\\
		a(h_{rx})=(1.1log(f_c)-0.7)h_{rx}-(1.56log(f_c)-0.8)
	\end{gather*}
	Con $f_c$ en MHz, $h$ en m y $d$ en km
	\textbf{1.} Calcule la potencia recibida en el móvil. Puede evaluar las pérdidas mediante la fórmula de Okumura-Hata\\
	\begin{gather*}
		a(h_{rx})=21.68\\
		L_{50urban}(dB)=151.54dB\\
		L_{50}(dB)=141.46dB\\
		P_{rx}=PIRE+G_{MS}-L_{50}=-81.46dB=7.14mW
	\end{gather*}
	\textbf{2.} Si se requiere una relación de protección frente a la interferencia cocanal de 18 dB y la estación base de la celda cocanal más cercana (con idénticos parámetros a los de la estación base bajo estudio) se encuentra a una distancia de 200 km, determine cuál es el margen de seguridad en dB.
\end{exercise}
% subsection ejercicios4 (end)