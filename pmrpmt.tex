\section{Redes de comunicaciones moviles terrestres}
\subsection{Sistemas PMR y PMT}
\begin{itemize}
\item{PMR(Private Movil Radio):} Sistemas normalmente no conectados a la red telefónica pública conmutada que se modelan como sistemas de espera.
\item{PMT(Personal Movil Telecommunications):} Sistemas conectados a la red telefónica pública modelados como sistemas con pérdidas. 
\end{itemize}
\subsubsection{Private Movil Radio}
\label{ssub:PMR}
Las características que definen este sistema son las siguientes:
\begin{itemize}
	\item Tienen una area territorial limitada.
	\item No suelen estar conectados a la red telefónica pública conmutada(POTS).
	\item Se suelen usar para dar servicios de empresa como la gestión de flotas.
	\item Deben ser posibles tanto las llamadas entre estaciones moviles (MS) entre sí como las llamadas a grupos.
	\item Deben funcionar en regimen de espera con llamadas frecuentes y de corta duración.
	\item Lo normal es que funcionen en simplex pero hay casos de sistemas en semiduplex e incluso en full-duplex.
\end{itemize}
El ejemplo más representativo de los sistemas PMR es el sistema Trans European Trunked RAdio (TETRA). Dependiendo de como se asignen los canales de comunicación a los ususarios del sistema se puede distinguir entre los dos siguientes sistemas:
\begin{itemize}
	\item Asignación rígida: a un conjunto de ususarios se les asigna un único canal para la comunicación. Es un sistema de asignación bastante poco efectivo, por esto, se suele usar con colectivos relativamente pequeños.
	\item Asignación troncal: se tienen N canales que pueden ser usados por M usuarios.
\end{itemize}
l
\begin{example}[Sistemas PMR]
Se tiene un sistema PMR en el que los usuarios piden 1 llamada de 20 segundos por HC
\begin{gather*}
	\mu=\sfrac{1}{20}s^{-1}\\
	a_o=\frac{\lambda}{\mu}=\frac{20}{3600}=5.56mE	
\end{gather*}
\end{example}
% subsubsection PMR (end)