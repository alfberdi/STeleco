\section{Sistemas de Radiocomunicación por Satélite}
\label{sec:satelite}
	\subsubsection{Introducción e historia}
	\label{ssub:introSat}
		El objetivo delos sistemas de radiocomunicación por satelite es el establecimiento de enlaces entre estaciones, tanto fijas como móviles, mediante repetidores en la órbita terrestre. Este concepto se refleja ya en artículos desde 1929. Los repetidores pueden ser tanto activos como pasivos, es decir, pueden amplificar la señal o no.\\
		La carrera espacial empieza con la idea de utilizar el espacio para lo militar. En 1957 se lanzan los primeros Sputnik y a Laika. Mientras tanto en estados unidos crean la NASA, sin lanzar nada al espacio. A partir de 1958 se empiezan a lanzar satelites de comunicación de la NASA con el departamento de defensa yanki. El primero fue un repetidor pasivo. En 1963 se pone en orbita el primer satélite geoestacionario, el Syncom 2.\\
		En 1964 se crea INTELSAT, una organización de 11 paises para la creación de un sistema comercial mundial de telecomunicaciones vía satélite. En la actualidad es privado y lo componen 109 paises. El primer satélite fue el early bird (INTELSAT I).\\
		Los sistemas satélite son una alternativa a otros sistemas. Con 3 satélites geoestacionarios se puede cubrir toda la superficie terrestre. Los equipos de comunicación y control deben ser muy fiables, ya que, la reparación puede ser imposible, además de estar expuestos a gran cantidad de radiaciones. Tienen una vida util limitada por el fin del combustible usado para los motores, una vez sin combustible no se pueden reorientar y se envian a una órbita basura. Los sistemas satélite se suelen usar como complemento, por ejemplo, un cable submarino puede tener un sistema satélite de soporte.
	% subsubsection introSat (end)
	\subsubsection{Servicios de satélites}
	\label{ssub:serviciosSat}
		\begin{itemize}
			\item Servicio fijo: Enlaces entre dos puntos terrestres usando el satélite como repetidor.
			\item Servicio Móvil: Uno o más puntos fijos y móviles, como barcos, aviones y torres de control.
			\item Servicio de radiodifusión: Uno o más puntos fijos y terminales dispersos, como la televisión o la radio satélite.
			\item Servicio de radiodeterminación: Localización determinada por satélites, como los sistemas GPS o Galileo.
			\item Servicio de exploración de la tierra: desde mapas del tiempo (meteosat), creación de mapas hasta la exploración de recursos.
			\item Servicio de exploración del espacio: mediante el uso de telescopios o radiotelescopios como el Hubble.
			\item Servicio entre satélites: La comunicación entre satélites no tiene limitaciones físicas para el uso de ancho de banda, por ejemplo, a 60GHz el oxígeno absorbe la señal, en el espacio no.
		\end{itemize}
	% subsubsection serviciosSat (end)
	\subsubsection{Estructura del sistema de comunicación satélite}
	\label{ssub:estructSat}
		\begin{figure}[htp]
			\centering
			\includegraphics[width=0.7\textwidth]{Imagen/arquisatelite.jpg}
			\caption{Estructura de un sistema de comunicación satélite}
		\end{figure}
		\begin{itemize}
			\item Estación terrena de transmisión: Recibe la señal para posteriormente modularla a una radiofrecuencia (RF), amplificarla y después transmitirla. En transmisión se necesita mucha potencia y una gran directividad.
			\item Enlaces: tanto el enlace ascendente como el descendente se modelan como sistemas de transmisión en espacio libre con pérdidas ocasionadas por la frecuencia, distancia, atmosfera e incluso la lluvia.
			\item Satélite: Se trata de una estación repetidora, que amplifica, cambia de banda y retransmite las señales. Las partes involucradas en cada acción vienen descritas a continuación:
			\begin{itemize}
				\item Recepción: La antena, un filtro y un amplificador de bajo ruido.
				\item Transpondedor: Se trata de un conversor de frecuencia y amplificador encargado de llevar a cabo la retransmisión.
				\item Conmutación: piezas encargadas del encaminamiento y por tanto de la asignación de transpondedores.
				\item Transmisión: Amplificación, filtrado y antena de transmisión.
			\end{itemize}
			\item Estación terrena de recepción: Hace uso de un recptor superheterodino, receptor de ondas de radio que utiliza un proceso de mezcla de frecuencias o heterodinación para convertir la señal recibida en una a frecuencia intermedia. Esta señal sin la radioportadora de alta frecuencia es mucho más facil de manejar que la original.
			\item Segmento espacial: La suma de los retardos en los enlaces, en transmisión, en recepción y en el satélites terminan produciendo grandes retardos, 250ms en geoestacionarios. Este hecho hace necesario el uso de técnicas de cancelación de ecos.
		\end{itemize}
		El diseño de estos sistemas de comunicación satélite tiene que tener en cuenta los siguientes aspectos:
		\begin{itemize}
			\item Órbita: En la mayoría de las ocasiones se utiliza una órbita geoestacionaria. Para comunicaciones móviles se empiezan a utilizar órbitas meo y leo para reducir el retardo y la potencia transmitida.
			\item Cobertura: Se pueden usar varios transpondedores para poder conformar un haz de diferente tipo y anchura.
			\item Conectividad: Capacidad de establecer enlaces entre estaciones terrenas. 
			\item Técnicas de acceso múltiple para la compartición del satélite. Se suelen utilizar FDMA y TDMA, en algunas ocasiones se puede utilizar CDMA pero supone un gran problema.
			\item Banda de frecuencia y ancho de banda: Se utilizan diferentes bandas según el servicio y la disponibilidad necesitada. Las bandas a mayor frecuencia han de transmitir más potencia para superar la peor propagación. Para aumentar la reutilización se separan haces de la misma frecuencia polarizandolos opuestos. Para aumentar aún más la eficiencia se mejoran el tratamiento de la señal y las modulaciones.
			\item Potencia: Se busca un compromiso entre la distancia a recorrer y la limitación a bordo del satélite. Una nueva limitación es la interferencia con otros satélites y estaciones terrenas.
		\end{itemize}
	% subsubsection estructSat (end)
	\subsubsection{Órbitas}
	\label{ssub:orbitas}
		Las óbitas se basan en las leyes de Kepler, consecuencia de la ley de gravitación universal. La elección de la órbita depende del tipo de cobertura, aplicación o recursos económicos. Las órbitas disponibles son escasas. Existen dos tipos de órbitas, las geoestacionarias y las oblícuas. Las órbitras geoestacionarias (GEO) son órbitas circulares, en una latitud baja, inclinación nula y un periodo de revolución igual al ciclo terrestre. Estos satélites siempre son visibles y no requieren, teoricamente, de un seguimiento. En las órbitas oblicuas, en cambio, el seguimiento ha de ser continuo, ya que el satélite sale y se pone. Dependiendo de la altura de la órbita puede ser: baja (LEO), media (MEO) o alta (HEO). Tanto los LEO como los MEO son más baratos de lanzar que los GEO.\\
		Las ventajas de las órbitas de tipo geoestacionaria son, la gran superficie de cobertura y permiten la existencia de estaciones terrenas fijas. Las mayores desventajas son la potencia necesaria para la transmisión, las antenas necesarias, los retardos, los lanzamientos de alto coste y la imposibilidad que presentan de cubrir latitudes altas.
		La órbita geoestacionaria viene definida por la siguiente formula, tomando los siguientes valores. $\omega=\frac{2\pi}{T}$ para un periodo de rotación T=23h 56min.
		\[G\frac{Mm}{d^2}=md\omega^2\]
		De esta formula y teniendo en cuenta un radio de la tierra de 6366km, se obtiene una distancia a la superficie de la tierra de 35806km.\\
		La cobertura del satélite tiene dos formas, la cobertura geométrica es la cobertura real que ofrece el satélite, esta es diferente de la cobertura radioeléctrica. La cobertura radioélectrica es igual a la geométrica pero tiene en cuenta el ruido terrestre y los obstáculos físicos, es decir es una cobertura real.
	% subsubsection orbitas (end)
	\subsubsection{Distancias satelitales}
	\label{ssub:distSatelite}
		Las coordenadas cartesianas del satélite son las siguientes: 
		\begin{gather*}
			x_s=R+h\\
			y_s=0\\
			z_s=0
		\end{gather*}
		Las coordenadas de la estación terrena, en cambio:
		\begin{gather*}
			x_e=Rcos\phi cos\lambda\\
			y_e=Rsin\phi cos\lambda\\
			z_e=Rsin\lambda
		\end{gather*}
		Los ángulos $\phi$ y $\lambda$ son longitud y latitud. $\phi$ se calcula como la diferencia entre la longitud del satélite y la longitud de la estación terrena. $\phi_l=\phi_s-\phi_e$. El ángulo $\lambda$ es la latitud de la estación terrena, ya que, la latitud del satélite, al estar sobre el ecuador es 0.\\
		La distancia entre el satélite y la estación terrena se puede calcular por medio de una simple regla de pitagoras:
		\[ES=d=\sqrt{(x_s-x_e)^2+y_e^2+z_e^2}=\sqrt{(R+h)^2+R^2-2R(R+h)cos\phi cos\lambda}\]
	% subsubsection distSatelite (end)
	\subsubsection{Balance de Enlace}
	\label{ssub:balance}
		Es el calculo de potencias que permite determinar la calidad de un enlace. El balance del enlace se puede generalizar con la siguiente formula.
		\[\frac{c}{N_0}=P_{tx}+G_{tx}+\frac{G_{rx}}{T}-L-BO-10logKB\]
		De la fórmula sabemos que T es la temperatura del receptor, K es la constante de Boltzman, B el ancho de banda del canal, L son las pérdidas del enlace, BO es el Back-Off. El back-Off se diferencia en: Input Back-Off para el sentido ascendente y Output Back-Off para el descendente, IBO y OBO. El factor de calidad del enlace es el cociente entre la ganancia y temperatura del receptor.\\
		La calidad del enlace para enlaces digitales se calcula de la siguiente forma:
		\[\frac{e_b}{n_0}=\frac{c}{n_0}\frac{1}{R_b}\]
	% subsubsection balance (end)
% section satelite (end)